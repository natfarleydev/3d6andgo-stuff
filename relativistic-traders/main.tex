\documentclass[a4paper,twocolumn]{memoir}

\usepackage[british]{babel}
\usepackage{libertine}
\usepackage{microtype}
\usepackage{gurps}
\usepackage{lettrine}

\renewcommand{\familydefault}{\sfdefault}

\begin{document}

\title{Relativistic traders}
\author{Nathanael Farley\thanks{\SJGamesOnlinePolicyGameAid{Nathanael Farley}}}
\maketitle

% In sci-fi settings where FTL travel is not possible, interstellar trade can be
% difficult. Traders who devote their lives to trading between stellar systems are
% called `Relativistic traders' because of the extent to which 

\lettrine{R}{elativistic} traders (RTraders) are a result of the demand for
interstellar trade without FTL travel. Since they regularly travel at speeds
close to the speed of light, they appear to age much slower than those on the
surface of planets who follow \emph{galactic time}. To the traders (who follow
\emph{ship time}) the trips feel much quicker. For most traders, the difference
is for every 10 years in galactic time 1 year passes in ship time. 
% TODO add a little maths here to be sure of what's happening

RTraders trade in unique goods such as new technologies and arts from other star
systems. Their trade is often in kind, eschewing formal currency as its value
can fluctuate too wildly between visits. They may bring items such as:
\begin{itemize}
\item medical breakthroughs
\item industrial techniques for new materials
\item work by a notable artist
\item information on other parts of the galaxy
\end{itemize}
The most lucrative work is commissioned work: companies or individuals can pay
many times the going rate for specific kinds of goods in order to give them an
advantage in their home market. \footnote{Without RTrade, there would be little
  sharing of technology between colonies apart from radio communications. On
  many colonies, no such communications are attempted because there is no profit
  in it and therefore no reason to set it up.}

Occasionally, traders are paid for `future travel' where the individual requires
an item to be transported to the future. This can be an artifact that needs
preservation for as long as possible, such as a religious artifact or a family
heirloom. Rarely, it may be a person who wants to avoid the law and outrun the
statute of limitations on a crime. Such travel is not \emph{illegal}, but the
authorities will obviously try to stop such a use of trader ships.

When approaching a known star system traders will usually send a message via
radio waves describing their cargo before they leave. This message arrives 6
months to a year before they do, so people will be interested in their cargo as
they arrive.

% 
When they arrive, there is no delay in the bartering process; the traders have
been travelling for a year (10 years in galactic time) and are ready for trade!
They usually stay in place for about a year as they trade for food, parts and
more unique goods to send to the next star system. Once they are satisfied with
their haul they send a message to the next system they are travelling to and
head out.


The communities that spring up around RTrade are often very insular since
friends that stay planetside age approximately 5x faster than ship travellers.
Large fleets of ships travel together for this reason; it is not unusual for
fleets to be a few generations old with their own distinct culture.

Trade ships are also very out of place at the spacedock. If a ship is designed
to last 100 years, then it is 500 years out of fashion by the time it is
decomissioned. For this reason, traders opt more often for modular ship designs
that can be updated regularly. RShip engineers are extremely adept at joining
systems that have no place together. If they were not, their ship would simply
cease to function properly!


RTraders are usually accepted in most rational societies. Since they provide
great value to those who can afford it, they are respected and well treated,
although reports of violence on planets that feel their culture is threatened by
new ideas (such as ideas expressed by arts goods frequently stored by traders)
is not uncommon. When met with violence, traders are spooked easily and fleet
commanders will often call for a vote to leave the system immediately instead of
fighting with trader weapons that are likely decades out of date. If the vote is
unanimous, all ships in that fleet will leave immediately even if there are
ongoing trade deals.


\section{Lenses}
\label{sec:lenses}

\subsection{Trader}
\label{sec:trader}

\begin{character}
  \IQ{12}
  \skill{Finance}{14}
\end{character}



\end{document}

%%% Local Variables:
%%% mode: latex
%%% TeX-master: t
%%% End:
