\documentclass{article}
\usepackage[british]{babel}

\usepackage{times}

\usepackage{geometry}
\usepackage{xparse}

\usepackage{gurps}
\usepackage{gurps-npccard}

\newcommand{\pokemon}{Pok\'emon\xspace}

\begin{document}
\title{Ash Ketchum from 3d6 and Go!}
\author{Nathanael Farley\thanks{\SJGamesOnlinePolicyGameAid{Nathanael Farley}}}
\maketitle

\getcharacterfromgcx*{ash_ketchum}{Ash Ketchum}
\getcharacterfromgcx*{pikachu}{Pikachu}

Ash Ketchum from the town of Pallet! \pokemon was a large part of my childhood
so it was good fun to design him in \gurps. This is mostly based on the TV show,
but it has a little bit of the game in it.

\section*{Designer's notes}
\label{sec:designers-notes}

Since there is no official \pokemon adaption for \gurps, designing Ash meant
defining how a \pokemon game could be played. Most interactions can be boiled
down to persuasion rolls (e.g.~`persuade' them to get into the pok\'eball to
capture them, etc.)

For this build, I added a couple of things. First, this wildcard skill:
\begin{quote}
  \begin{description}
  \item[Pokemon Trainer!] Use this skill in place of Animal Handling (All
    \pokemon), Expert Skill (\pokemon Trainer), rolls to persuade \pokemon and
    rolls to catch \pokemon.
  \end{description}
\end{quote}
Second, \emph{Reputation} as badges. \pokemon seem to magically recoginise a
trainer with badges (at least in the game) so if you have sufficient reputation
and manage to capture a \pokemon, then they will simply obey you. (This can be
ruled as +4 (Captured) to persuasion roll, but I would probably handwave it.)

\pokemon trainers in the original games had 6 \pokemon which they could carry
around at a time. I've modelled this as \emph{Modular Abilities (Chip Slots)}
where each `slot' is actually 8 points of \emph{Ally} (at least it's 8 points at
Ash's initial level).

\vfill
\begin{center}
  \textsc{Next page for NPC cards!}
\end{center}


\clearpage
\centering
\vspace*{\fill}
\printcharactercard{Ash Ketchum}
\vfill
\printcharactercard{Pikachu}
\vfill





% \SetCharacterKey{Saitama}
% \GCPrintCharacter
% \ResetCharacterKey
\end{document}

%%% Local Variables:
%%% mode: latex
%%% TeX-master: t
%%% End:
